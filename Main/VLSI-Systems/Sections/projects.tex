

\section{Projects}
    \resumeSubHeadingListStart
    
        

\resumeProjectHeading{\textbf{Battle Bot Communication System} $|$ \emph{C/C++, SPI, UART, Arduino, ESP32, nRF24L01, Git}}{\textbf{May 2024}}
      \resumeProjectDescription{Team project ($\sim$35 members) at CombatCraft, Toronto, ON}{\textbf{Developed and implemented} a long-range communication system for a battle bot and controller, leveraging the \textbf{nRF24L01} module and \textbf{ESP32} for \textbf{reliable radio-frequency communication}.}
      \resumeItemListStart
        \resumeItem{\textbf{Engineered} communication protocols to \textbf{improve transmission efficiency by 15\%}, reduce latency, and enhance data integrity over long distances.}
        \resumeItem{\textbf{Developed embedded C firmware} with layered SPI communication for data handling and protocol execution.}
        \resumeItem{\textbf{Integrated ESP32} for real-time data transmission to a remote server and \textbf{conducted comprehensive testing and validation} to ensure scalability and reliable communication over distances \textbf{up to 25 meters}.}
        % \resumeItem{\textbf{Conducted comprehensive testing and validation} to ensure scalability and consistent communication across extended distances up to \textbf{25 meters}.}
      \resumeItemListEnd


        
        

\resumeProjectHeading
              {\textbf{Rover Firmware Development} $|$ \emph{C/C++, Arduino, ESP32/8266, Teensy, PlatformIO, Python, Git}}{\textbf{Mar. 2024}}
          \resumeProjectDescription{Team project ($\sim$75 members) at Ryerson Rams Robotics, Toronto, ON}{\textbf{Developed and optimized} embedded firmware for a Mars rover, interfacing \textbf{15+ sensors} (soil, air, water, temperature, pressure, UV, infrared) and \textbf{10+ motors} using \textbf{AVR} based microcontrollers for real-time operations.}
          \resumeItemListStart
            % \resumeItem{Interfaced multiple sensors (soil, air, water, temperature, pressure, UV, infrared) and motors with ESP32, Teensy, and Arduino microcontrollers, developing and optimizing embedded C/C++ firmware for real-time operations; refactored code to reduce memory usage and improve performance using algorithmic enhancements.}
            \resumeItem{\textbf{Refactored code} to reduce memory usage and improve performance using algorithmic enhancements, \textbf{implemented ROS (Robot Operating System)} for sensor integration and communication, and designed multi-sensor data processing for dynamic rover control. Also \textbf{implemented communication protocols} to transmit real-time sensor data and GPS coordinates for live monitoring via a GUI.}
            % \resumeItem{Designed and integrated multi-sensor data processing for dynamic rover control, receiving commands via radio from the controller and remote server; implemented communication protocols to transmit real-time sensor data and GPS coordinates to a remote server for live monitoring via a GUI.}
          \resumeItemListEnd



        

\resumeProjectHeading
          {\textbf{8-Bit Custom CPU Design \& FPGA Implementation} $|$ \emph{VHDL, FPGA, Quartus II, ModelSim, Git}}{\textbf{Nov. 2023}}
          \resumeProjectDescription{Individual project at university}{\textbf{Designed and implemented} a custom 8-bit CPU architecture in \textbf{VHDL}, featuring an ALU and Control Unit (FSM and 4:16 Decoder) to perform arithmetic, logic (AND, OR, XOR), and bit-shifting operations.}
          % \resumeItem{\emph{Individual project at university}}
          % \vspace{-0.25cm}
          % \resumeItem{Designed and implemented a custom 8-bit CPU architecture using VHDL, focusing on executing arithmetic and logic operations.}
          \resumeItemListStart
            % \resumeItem{Engineered a custom CPU architecture using VHDL, integrating a robust Arithmetic Logic Unit (ALU), Control Unit (Finite State Machine and Decoder), and I/O registers to execute complex arithmetic and logic operations on 8-bit inputs.}
            % \resumeItem{Developed a high-performance ALU supporting operations such as addition, subtraction, bitwise manipulations (AND, OR, XOR), and advanced functions like bit shifts and rotations, controlled via a state-driven FSM and 4:16 decoder.}
            % \resumeItem{Simulated, verified and optimized the CPU’s operation in Quartus, utilizing waveform analysis (timing analysis) to validate accurate real-time processing of input signals, control opcodes, and clock cycles before FPGA synthesis.}
            % \resumeItem{Synthesized and implemented the CPU on an FPGA, using external inputs and seven-segment displays to visualize real-time outputs, successfully demonstrating the integration and operation of all components.}
            % \resumeItem{Designed and implemented a robust ALU and Control Unit (FSM and 4:16 Decoder) to perform 8-bit operations, including addition, subtraction, bitwise logic (AND, OR, XOR), bit shifts, and rotations.}
            \resumeItem{\textbf{Simulated}, \textbf{verified}, \textbf{optimized}, and \textbf{synthesized} CPU performance in \textbf{Quartus} using \textbf{ModelSim} for waveform analysis. \textbf{Optimized the FSM} in the Control Unit with \textbf{state minimization}, increasing clock frequency by \textbf{10\%} and opcode execution by \textbf{25\%}. \textbf{Implemented} on \textbf{FPGA} with real-time output on seven-segment displays.}
          \resumeItemListEnd



        

\resumeProjectHeading
          {\textbf{Intelligent Home Energy Optimization System} $|$ \emph{C/C++, Arduino, Python, PlatformIO, Git}}{\textbf{Feb. 2023}}
          \resumeProjectDescription{Team Project (2 members)}{\textbf{Developed} an intelligent home energy system using a \textbf{Raspberry Pi 3B/4B} and \textbf{Arduino Uno R3/Nano} to \textbf{automate real-time}, energy-efficient control of lighting and computer display systems based on room occupancy.}
          \resumeItemListStart
            % \resumeItem{Developed and implemented an intelligent home energy optimization system leveraging a Raspberry Pi 3B and Arduino Uno R3/Nano, seamlessly integrating motion sensors and relays to automate energy-efficient control of lighting and multiple display systems.}
            % \resumeItem{Engineered a seamless integration of motion sensors \& relays with microcontrollers, enabling real-time, occupancy-based automation of lighting and appliances for optimized energy usage and dynamic environmental control.}
            % \resumeItem{Designed and implemented a low-power, efficient circuitry powered by a 5\emph{V}, 1\emph{A} source, ensuring consistent, reliable performance across all system components.}
            % \resumeItem{Achieved a 64.6\% reduction in electricity consumption for both lighting and monitor displays, showcasing significant energy savings through innovative automation and intelligent system design.}
            % \resumeItem{Interfaced and integrated motion sensors and relays with microcontrollers, enabling real-time, occupancy-based automation of appliances to optimize energy usage and environmental control.}
            \resumeItem{\textbf{Interfaced and integrated} motion sensors and EM relays with microcontrollers, \textbf{designed low-power circuitry} powered by a 5V, 1A source, achieving a \textbf{64.6\% reduction} in electricity consumption for lighting and computer display systems.}
          \resumeItemListEnd


        
      %     \resumeProjectHeading
      %     {\textbf{Firmware Engineer} $|$ \emph{C/C++, Arduino, ESP32/8266, Teensy, PlatformIO, Python, Git}}{\textbf{Mar. 2024}}
      % % {Ryerson Rams Robotics (Engineering Design Team)}{Toronto, ON}
      % \resumeProjectDescription{Team project ($\sim$75 members) at Ryerson Rams Robotics, Toronto, ON}{Developed and optimized embedded firmware for a Mars rover and custom controller, interfacing 15+ sensors and motors for real-time operations and remote monitoring.}
      % \resumeItemListStart
      %   % \resumeItem{Engineered and optimized embedded firmware in C/C++ for a Mars rover and custom controller, interfacing with a diverse array of sensors (soil, air, water, temperature, pressure, UV, infrared, etc.) to collect, process, and trigger data-driven operations in real-time.}
      %   % \resumeItem{Integrated multi-sensor data processing to dynamically control rover operations based on environmental readings while receiving movement commands via radio from both the controller and remote server for seamless navigation and task execution.}
      %   % \resumeItem{Designed and implemented advanced communication protocols to transmit real-time sensor data and GPS coordinates from the Mars rover to a remote server, enabling live monitoring through an intuitive, aesthetically designed GUI.}
      %   % \resumeItem{Deployed and programmed multiple Teensy and AVR microcontrollers, ensuring coordinated operation of sensors, motors, and communication modules while addressing complex wiring requirements and maintaining optimal system performance within tight resource constraints.}
      %   % \resumeItem{Enhanced system efficiency through advanced memory management and algorithmic optimization, ensuring maximum computational throughput and real-time performance for sensor-driven operations and bidirectional communication with external systems.}
      %   \resumeItem{Interfaced multiple sensors (soil, air, water, temperature, pressure, UV, infrared) and motors with ESP32, Teensy, and Arduino microcontrollers, developing and optimizing embedded C/C++ firmware for real-time operations; refactored code to reduce memory usage and improve performance using algorithmic enhancements.}
      %   \resumeItem{Designed and integrated multi-sensor data processing for dynamic rover control, receiving commands via radio from the controller and remote server; implemented communication protocols to transmit real-time sensor data and GPS coordinates to a remote server for live monitoring via a GUI.}
      % \resumeItemListEnd
      % \resumeProjectHeading{\textbf{Battle Bot Communication System} $|$ \emph{C/C++, Arduino, ESP32, nRF24L01, Git}}{\textbf{May 2024}}
      % \resumeProjectDescription{Team project ($\sim$35 members) at CombatCraft, Toronto, ON}{\textbf{Developed and implemented} a long-range communication system for a battle bot and controller, leveraging the nRF24L01 module and ESP32 for reliable radio-frequency communication.}
      % \resumeItemListStart
      %   \resumeItem{\textbf{Engineered} communication protocols to \textbf{improve transmission efficiency by 15\%}, reduce latency, and enhance data integrity over long distances. Integrated ESP32 for real-time data transmission to a remote server, enabling seamless performance monitoring and diagnostics.}
      %   \resumeItem{\textbf{Conducted comprehensive testing and validation} to ensure scalability and consistent communication across extended distances up to \textbf{25 meters}.}
      % \resumeItemListEnd
    \resumeSubHeadingListEnd

