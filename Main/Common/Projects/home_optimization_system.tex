

\resumeProjectHeading
          {\textbf{Intelligent Home Energy Optimization System} $|$ \emph{C/C++, Arduino, Python, PlatformIO, Git}}{\textbf{Feb. 2023}}
          \resumeProjectDescription{Team Project (2 members)}{\textbf{Developed} an intelligent home energy system using a \textbf{Raspberry Pi 3B/4B} and \textbf{Arduino Uno R3/Nano} to \textbf{automate real-time}, energy-efficient control of lighting and computer display systems based on room occupancy.}
          \resumeItemListStart
            % \resumeItem{Developed and implemented an intelligent home energy optimization system leveraging a Raspberry Pi 3B and Arduino Uno R3/Nano, seamlessly integrating motion sensors and relays to automate energy-efficient control of lighting and multiple display systems.}
            % \resumeItem{Engineered a seamless integration of motion sensors \& relays with microcontrollers, enabling real-time, occupancy-based automation of lighting and appliances for optimized energy usage and dynamic environmental control.}
            % \resumeItem{Designed and implemented a low-power, efficient circuitry powered by a 5\emph{V}, 1\emph{A} source, ensuring consistent, reliable performance across all system components.}
            % \resumeItem{Achieved a 64.6\% reduction in electricity consumption for both lighting and monitor displays, showcasing significant energy savings through innovative automation and intelligent system design.}
            % \resumeItem{Interfaced and integrated motion sensors and relays with microcontrollers, enabling real-time, occupancy-based automation of appliances to optimize energy usage and environmental control.}
            \resumeItem{\textbf{Interfaced and integrated} motion sensors and EM relays with microcontrollers, \textbf{designed low-power circuitry} powered by a 5V, 1A source, achieving a \textbf{64.6\% reduction} in electricity consumption for lighting and computer display systems.}
          \resumeItemListEnd

